% -- coding: UTF8 --
\documentclass[UTF8]{ctexart}

\usepackage{amsmath}
\title{低观点下的线性微分方程的解法}
\author{miroox}
\date{\today}

\newcommand\mathe{\mathit{e}}

\begin{document}

\maketitle
\begin{abstract} 
一种不涉及线性微分方程解的结构的方法。
\end{abstract} 

\section{一阶常系数线性非齐次微分方程的另解} \label{sec:1ln}
	对于非齐次方程
	\begin{equation}
		y'=p(x)y+q(x),q(x)\neq0 \label{eq:1ln-org}
	\end{equation}
	即
	\begin{equation}
		y'-p(x)y=q(x) \label{eq:1ln-mov}
	\end{equation}
	为了在上式左边得到导数乘法公式的形式 
	\begin{equation}
		(f\cdot g)'=f'\cdot g+f\cdot g' \label{eq:mult}
	\end{equation}
	等式两边同乘 $f(x)$ ,即
	\begin{equation}
		y'\cdot f(x)-p(x)y\cdot f(x)=q(x)f(x) \label{eq:1ln-mult}
	\end{equation}
	比对公式(\ref{eq:mult}),$f(x)$ 应满足 $f'(x)=-p(x)f(x)$
	容易解得
	\begin{equation}
		f(x)=\mathe^{-\int p(x)dx} \label{eq:1ln-fx}
	\end{equation}
	(任意常数显然没有必要) \\
	则(\ref{eq:1ln-mult})式可以化成 $(y\cdot f(x))'=q(x)f(x)$,解得
	\begin{equation}
		y=\frac{1}{f(x)}(C+\int q(x)f(x)dx) \label{eq:1ln-yNoExpand}
	\end{equation}
	再带入(\ref{eq:1ln-fx}),即有
	\begin{equation}
		y=\mathe^{\int p(x)dx}(C+\int q(x)\mathe^{-\int p(x)dx}dx) \label{eq:1ln-result}
	\end{equation}
	本方法的核心在于凑出导数乘法公式形式,相当于再解一个齐次微分方程,思路比常数变易法要显然,因此也不必记忆公式,知道凑乘法的思路即可,而且容易推广到其它问题上(比如中值定理凑函数的问题,以及下面的高阶线性微分方程等)。

\section{二阶常系数线性齐次方程 $y''+py'+qy=0$ } \label{sec:2lh}
	令 $p=-(r_{1}+r_{2})$, $q=r_{1}r_{2}$,即 $r_{1}, r_{2}$ 是 $x^{2}+px+q=0$ 的根,显然存在。则
	\begin{align}
		y''-(r_{1}+r_{2})y'+r_{1}r_{2}y&=0 \label{eq:2lh-repr} \\
		\Rightarrow (y''-r_{1}y')-r_{2}(y'-r_{1}y)&=0 \\
		\Rightarrow (y'-r_{1}y)'-r_{2}(y'-r_{1}y)&=0
	\end{align}
	可得
	\begin{equation}
		y'-r_{1}y=C_{0}\mathe^{r_{2}x} \label{eq:2lh-1stInt}
	\end{equation}
	\subsection{若 $r_{1}\neq r_{2}$ } 
		对(\ref{eq:2lh-1stInt})式两边同时乘 $\mathe^{-r_{1}x}$ ,则有
		\begin{align}
			\mathe^{-r_{1}x}y'-r_{1}\mathe^{-r_{1}x}y&=C_{0}\mathe^{(r_{2}-r_{1})x} \\
			\Rightarrow (\mathe^{-r_{1}x}y)'&=C_{0}\mathe^{(r_{2}-r_{1})x} \\
			\Rightarrow y&=\frac{C_{0}}{r_{2}-r_{1}}\mathe^{r_{2}x}+C_{1}\mathe^{r_{1}x} \label{eq:2lh-2ndInt1}
		\end{align}
		不妨记 $C_{2}=\frac{C_{0}}{r_{2}-r_{1}}$ ,即
		\begin{equation}
			y=C_{2}\mathe^{r_{2}x}+C_{1}\mathe^{r_{1}x} \label{eq:2lh-result1}
		\end{equation}
	\subsection{若 $r_{1}=r_{2}$ ,记作r}
		对(\ref{eq:2lh-1stInt})式两边同时乘 $\mathe^{-rx}$ ,则
		\begin{align}
			\mathe^{-rx}y'-r\mathe^{-rx}y&=C_{0} \\
			\Rightarrow (\mathe^{-rx}y)'&=C_{0} \\
			\Rightarrow y&=C_{0}x\mathe^{rx}+C_{1}\mathe^{rx} \label{eq:2lh-2ndInt2}
		\end{align}
		事实上,对(\ref{eq:2lh-2ndInt1})式取极限 $r_{2}\rightarrow r_{1}=r$ 亦可得到(\ref{eq:2lh-2ndInt2})式。

\end{document}